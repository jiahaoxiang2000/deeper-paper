%%%% IACR Transactions TEMPLATE %%%%
% This file shows how to use the iacrtrans class to write a paper.
% Written by Gaetan Leurent gaetan.leurent@inria.fr (2020)
% Public Domain (CC0)

%%%% 1. DOCUMENTCLASS %%%%
\documentclass[journal=tches,final]{iacrtrans}
%%%% NOTES:
% - Change "journal=tosc" to "journal=tches" if needed
% - Change "submission" to "final" for final version
% - Add "spthm" for LNCS-like theorems

%%%% 3. AUTHOR, INSTITUTE %%%%
\author{Jiahao Xiang\inst{1} \and Lang Li\inst{1}}
\institute{
  Hengyang Normal University, College of Computer Science and Technology, Hengyang, China
  % \and
  % Institute B, City, Country, \email{john@institute}
}
%%%% NOTES:
% - We need a city name for indexation purpose, even if it is redundant
%   (eg: University of Atlantis, Atlantis, Atlantis)
% - \inst{} can be omitted if there is a single institute,
%   or exactly one institute per author

%%%% 4. TITLE %%%%
\title[ML-DSA Digital Signatures in Resource-Constrained MQTT Environments]{ML-DSA Digital Signatures in Resource-Constrained MQTT Environments}
%%%% NOTES:
% - If the title is too long, or includes special macro, please
%   provide a "running title" as optional argument: \title[Short]{Long}
% - You can provide an optional subtitle with \subtitle.

\begin{document}

\maketitle

%%%% 5. KEYWORDS %%%%
\keywords{Post-Quantum Cryptography \and ML-DSA \and MQTT Protocol \and IoT Security \and Resource-Constrained Devices}

%%%% 6. ABSTRACT %%%%
\begin{abstract}
  The imminent threat of large-scale quantum computers necessitates the migration of Internet of Things (IoT) systems to post-quantum cryptographic standards. While NIST has standardized ML-DSA (Module-Lattice-Based Digital Signature Algorithm) for digital signatures, the practical deployment of post-quantum authentication in resource-constrained IoT environments remains unexplored. This research evaluates ML-DSA integration within MQTT-based IoT systems through comprehensive performance analysis on ARM Cortex-M4 microcontrollers. Our methodology encompasses signature generation and verification benchmarking, memory utilization analysis, and protocol overhead assessment under realistic IoT constraints. We analyze the performance implications of ML-DSA deployment compared to classical signature schemes, examining computational overhead, memory requirements, and verification latency on resource-constrained devices. These findings will reveal fundamental trade-offs between post-quantum security and IoT performance requirements, providing critical insights for practical deployment strategies in resource-limited environments.
\end{abstract}

\color{blue}

%%%% 7. PAPER CONTENT %%%%
\section{Introduction}

The emergence of quantum computing poses an existential threat to current cryptographic infrastructures, necessitating systematic migration to post-quantum cryptographic standards across all computing domains. The National Institute of Standards and Technology (NIST) has formalized ML-DSA (Module-Lattice-Based Digital Signature Algorithm) within FIPS 204~\cite{NIST-FIPS-204}, establishing this CRYSTALS-Dilithium-based scheme as the primary standard for post-quantum digital signatures.

While post-quantum standardization represents significant theoretical progress, practical deployment encounters severe constraints in resource-limited environments. Internet of Things (IoT) systems exemplify these challenges, where computational, memory, and energy limitations fundamentally constrain cryptographic implementation choices. The MQTT protocol, widely adopted for IoT messaging due to its lightweight characteristics, becomes particularly problematic when post-quantum signatures impose prohibitive performance overhead on resource-constrained devices. This disparity creates a critical gap between standardization achievements and practical deployment feasibility in IoT environments.

This research systematically addresses the challenge of ML-DSA integration within MQTT-based IoT systems through comprehensive performance analysis on ARM Cortex-M4 microcontrollers. Our investigation encompasses three critical dimensions: signature operation benchmarking, memory utilization analysis, and protocol overhead assessment under realistic IoT constraints. Through comparative analysis with classical signature schemes, we quantify computational overhead, memory requirements, and verification latency to establish fundamental trade-offs between post-quantum security guarantees and IoT performance requirements.

The remainder of this paper is organized as follows: Section~\ref{sec:related} presents background information on post-quantum cryptography and related work in IoT deployments. Section~\ref{sec:mldsa} provides an overview of the ML-DSA algorithm and its implementation considerations. Section~\ref{sec:architecture} describes our implementation architecture for MQTT-based IoT systems. Section~\ref{sec:methodology} details our experimental methodology for performance evaluation on ARM Cortex-M4 microcontrollers. Section~\ref{sec:results} presents and analyzes our experimental results, examining the trade-offs between security and performance. Finally, Section~\ref{sec:conclusion} concludes with implications for practical deployment and future research directions.

\section{Related Work and Motivation}\label{sec:related}

The transition from theoretical post-quantum standardization to practical deployment has revealed fundamental implementation challenges that extend beyond algorithmic considerations. While conventional network protocols have undergone extensive analysis for post-quantum migration~\cite{Kampanakis2020, Sikeridis2020}, IoT-specific communication protocols remain inadequately addressed, creating deployment barriers in resource-constrained environments.

\subsection{Post-Quantum Signature Performance in Embedded Systems}

Empirical evaluations consistently demonstrate that post-quantum signature schemes impose substantial computational overhead on ARM Cortex-M microcontrollers commonly deployed in IoT devices~\cite{Banegas2021, Marchsreiter2024}. These performance implications manifest across multiple operational dimensions.

Signature size analysis reveals substantial increases for post-quantum schemes. ML-DSA signatures range from 2,420 bytes (Level 1) to 4,595 bytes (Level 5) compared to 64 bytes for ECDSA, representing 30-70× size increases. These expanded signature sizes, combined with elevated computational demands, frequently exceed the processing capabilities of resource-constrained devices.

Banegas et al.~\cite{Banegas2021} quantified these performance implications through comprehensive benchmarking on embedded systems, establishing that CRYSTALS-Dilithium signature operations require approximately 45\% additional computational cycles compared to classical ECDSA implementations on ARM Cortex-M4 processors. This computational overhead compounds with memory constraints to create deployment bottlenecks in IoT environments.

\subsection{Deployment Bottlenecks in IoT Applications}

Practical deployment scenarios reveal critical performance bottlenecks that challenge IoT system viability. Analysis of the SUIT (Software Update for the Internet of Things) framework demonstrates that post-quantum signature verification operations require up to 3.2 seconds on low-power microcontrollers, substantially exceeding acceptable latency constraints for real-time IoT applications.

These performance constraints are compounded by security vulnerabilities in embedded implementations. Fault injection research targeting ML-DSA and ML-KEM implementations achieved 89.5\% attack success rates on ARM Cortex-M processors through electromagnetic fault injection techniques~\cite{Li2024}. The analyses demonstrate that Keccak-based hash functions—integral to ML-DSA randomness generation and signature computation—exhibit particular susceptibility to loop-abort faults enabling complete private key recovery, necessitating additional countermeasures that further impact performance.

\subsection{Alternative Approaches and Limitations}

Recent research has explored alternative authentication architectures to address these deployment challenges. Kim and Seo~\cite{Kim2025} demonstrate that direct application of post-quantum signatures to MQTT authentication introduces prohibitive performance overhead, prompting KEM-based authentication architectures that eliminate signature operations entirely. While their CRYSTALS-Kyber implementation achieves 4.32-second handshake completion on 8-bit AVR microcontrollers, this approach circumvents rather than resolves the fundamental challenge of post-quantum signature deployment.

Signature-based authentication mechanisms remain essential for applications requiring cryptographic non-repudiation, comprehensive audit trails, and compatibility with existing public key infrastructure frameworks. Consequently, optimization approaches targeting signature operations become critical for IoT deployment feasibility.

Current algorithmic optimization research reveals limitations in addressing fundamental resource constraints. Barrett multiplication techniques achieve 1.38-1.51× performance improvements on ARM Cortex-M3 processors and 6.37-7.27× improvements on 8-bit AVR platforms~\cite{Barrett2023}. However, these optimizations provide insufficient performance gains to bridge the gap between post-quantum signature requirements and IoT device capabilities, necessitating comprehensive system-level analysis.

\subsection{Research Gap and Motivation}

The absence of comprehensive empirical studies specifically evaluating ML-DSA performance within MQTT protocol implementations represents a critical knowledge gap in post-quantum IoT deployment. This gap becomes particularly significant given MQTT's widespread adoption in industrial IoT deployments, where signature-based authentication remains mandatory for regulatory compliance and security audit requirements.

Existing research primarily focuses on isolated cryptographic operations or alternative protocol architectures, failing to address the systematic integration challenges inherent in MQTT-based IoT systems. This research addresses these limitations through comprehensive performance analysis of ML-DSA deployment within realistic MQTT environments, providing essential insights for practical post-quantum IoT migration strategies.

\color{black}

\section{ML-DSA Algorithm Overview}\label{sec:mldsa}

\section{Implementation Architecture}\label{sec:architecture}

\section{Experimental Methodology}\label{sec:methodology}

\section{Results and Analysis}\label{sec:results}

\section{Conclusion}\label{sec:conclusion}

\newpage

%%%% 8. BIBLIOGRAPHY %%%%
\bibliographystyle{alpha}
\bibliography{abbrev3,biblio}
%%%% NOTES
% - Download abbrev3.bib and crypto.bib from https://cryptobib.di.ens.fr/
% - Use biblio.bib for additional references not in the cryptobib database.
%   If possible, take them from DBLP.

\end{document}
