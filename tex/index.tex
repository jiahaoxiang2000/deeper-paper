%%%% IACR Transactions TEMPLATE %%%%
% This file shows how to use the iacrtrans class to write a paper.
% Written by Gaetan Leurent gaetan.leurent@inria.fr (2020)
% Public Domain (CC0)


%%%% 1. DOCUMENTCLASS %%%%
\documentclass[journal=tches,final]{iacrtrans}
%%%% NOTES:
% - Change "journal=tosc" to "journal=tches" if needed
% - Change "submission" to "final" for final version
% - Add "spthm" for LNCS-like theorems


%%%% 2. PACKAGES %%%%
\usepackage{lipsum} % Example package -- can be removed


%%%% 3. AUTHOR, INSTITUTE %%%%
\author{Jiahao Xiang\inst{1} \and Lang Li\inst{1}}
\institute{
  Hengyang Normal University, College of Computer Science and Technology, Hengyang, China
  % \and
  % Institute B, City, Country, \email{john@institute}
}
%%%% NOTES:
% - We need a city name for indexation purpose, even if it is redundant
%   (eg: University of Atlantis, Atlantis, Atlantis)
% - \inst{} can be omitted if there is a single institute,
%   or exactly one institute per author


%%%% 4. TITLE %%%%
\title{Fault Attacks}
%%%% NOTES:
% - If the title is too long, or includes special macro, please
%   provide a "running title" as optional argument: \title[Short]{Long}
% - You can provide an optional subtitle with \subtitle.

\begin{document}

\maketitle


%%%% 5. KEYWORDS %%%%
\keywords{Fault attack }


%%%% 6. ABSTRACT %%%%
\begin{abstract}
 TODO: Write an abstract
\end{abstract}


%%%% 7. PAPER CONTENT %%%%
\section{Introduction}

Widely used primitives like the AES~\cite{AES} do not have perfect
security, and can be analysed with linear
cryptanalysis~\cite{AC:DLNS20}, differential cryptanalysis, and a variety of implementation attacks. 
Among these, fault attacks have emerged as a powerful class of physical attacks that exploit hardware vulnerabilities to induce errors during cryptographic computations. 
Since the first demonstration of fault attacks on AES, numerous techniques have been developed to recover secret keys by injecting faults and analyzing the resulting faulty ciphertexts~\cite{TCHES:ZZJZBZ20,EC:SBRPM20}. 
Persistent fault attacks, for example, can compromise the security of AES implementations even in the presence of countermeasures~\cite{TCHES:ZZJZBZ20}. 
To address these threats, researchers have proposed various protection mechanisms, such as statistical ineffective fault attack countermeasures~\cite{TCHES:DDEGMP20}. 
Despite these advances, fault attacks remain a significant concern for the practical security of AES and other symmetric ciphers, motivating ongoing research into both new attack strategies and robust defenses. 

\section{Main Result}
\label{sec:main}


%%%% 8. BILBIOGRAPHY %%%%
\bibliographystyle{alpha}
\bibliography{cryptobib/abbrev3,crypto_custom,biblio}
%%%% NOTES
% - Download abbrev3.bib and crypto.bib from https://cryptobib.di.ens.fr/
% - Use bilbio.bib for additional references not in the cryptobib database.
%   If possible, take them from DBLP.

\end{document}
